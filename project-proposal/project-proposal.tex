\documentclass{article}

% Frequently Asked Questions, Notes, ....
% 01) You can use Czech or English to fill in the template.
% 02) Fill in the metadata section.
%

\usepackage[a-2u]{pdfx}
\usepackage[utf8]{inputenc}
\usepackage{array}
\usepackage[ddmmyyyy]{datetime}
\usepackage{geometry}
\usepackage[skip=7pt, indent=25pt]{parskip}

\renewcommand{\dateseparator}{.}
\renewcommand{\baselinestretch}{1.5}

\geometry{margin=2cm}

\def\ResearchProject{Research project}
\def\CompanyProject{Company project}
\def\SoftwareProject{Software project}

% METADATA SECTION START
\def\StudentsFullName{Bc. Ivona Oboňová}
\def\Obor{Computer Science - Software and Data Engineering}
\def\ProjectTitle{Extraction of GroupRule component into standalone service}
% Use "\ResearchProject", "\CompanyProject", \SoftwareProject"
\def\ProjectType{\CompanyProject}
\def\SupervisorFullName{Ing. Pavel Koupil, Ph.D.}
\def\ConsultantsFullName{Bc. Max Kovykov}
\def\ExpectedStart{2025-04-01}
\def\ExpectedEnd{2025-09-01}
% METADATA SECTION END

\title{\ProjectTitle}
\author{\StudentFullName}

\begin{document}

\centerline{\Large \textbf{Proposal of team software project}}
\centerline{ \textbf{Department of Software Engineering}}
\centerline{ \textbf{Faculty of Mathematics and Physics, Charles University}}
\bigskip
{\noindent\begin{tabular*}{\textwidth}{ >{\raggedleft}m{4cm} l}
 {\bf Solvers:} & \StudentsFullName \\
 {\bf Study program:} & \Obor \\
 & \\
 {\bf Project title:} & \ProjectTitle \\
 {\bf Project type:} & \ProjectType \\
 & \\
 {\bf Supervisor:} & \SupervisorFullName \\  
 {\bf Consultants:} & \ConsultantsFullName \\
 & \\
 {\bf Expected start:} & \ExpectedStart \\
 {\bf Expected end:} & \ExpectedEnd \\  
\end{tabular*}}

\section{Introduction}

At \textbf{Avast}, now part of \textbf{Gen Digital}, we specialize in cybersecurity software,
delivering advanced threat detection powered by AI, along with antivirus protection and privacy solutions
such as anti-tracking software, VPNs, and data breach monitoring. Our backend engineering team is
at the core of this mission, processing millions of malware samples daily and developing and
maintaining internal backend services and APIs that support our malware analysts, security
researchers, and other experts in enhancing our products.

One of our core systems, \textbf{Appserver}, is a .NET-based monolithic application running on
multiple Windows servers. It stores critical data and provides numerous APIs that analysts rely on
daily to identify and manage malware threats, making it an essential part of our infrastructure.
Among its many components, the \textbf{GroupRules} module is responsible for managing string-based
and multi-string-based malware definitions and rules. However, as Appserver has grown significantly
over time, it has become increasingly difficult to navigate, maintain, and extend with new
features.

To address these challenges, our team has begun decoupling the system into a service-based
architecture. As part of this initiative, this project focuses on migrating the GroupRules
component into a standalone service. This transition will enhance scalability, maintainability, and
flexibility. By isolating this component from the monolith, we aim to improve performance and
streamline malware definition management, creating a more efficient and future-proof solution.

\section{Project description}

As mentioned in the introduction, we aim to migrate the \textbf{GroupRules} component into a
standalone service, decoupling it from the \textbf{Appserver}, which has grown increasingly
complex, making it difficult to navigate and extend. Additionally, the GroupRules component is
embedded within a complex relational database schema. The current implementation uses a PostgreSQL
database to store data, which has a largely graph-like structure. As a result, retrieving data in
the required format and meeting specific conditions is both complex and inefficient, further
complicating maintenance and updates.

To address these issues, we will not only extract GroupRules into its own service, but also
evaluate whether a graph database would be a better fit for storing and managing string-based and
multi-string-based malware definitions. Given the interconnected nature of the data and its
relationships, graph databases appear to be a promising alternative. As part of this project, we
will assess their feasibility. If a graph database proves to be the optimal choice, we will
identify the best technology and implement GroupRules on top of it. If not, we will continue using
the existing relational database while refining its structure where possible.

Rather than rewriting the entire component, which would introduce significant complexity and risk,
we will focus on rebuilding only its core functionality as an independent service. This approach
allows for gradual integration into the existing infrastructure without disrupting public-facing
APIs and UIs. By maintaining unchanged external interfaces, we ensure a smooth and controlled
transition, enabling early validation of the new architecture.

\end{document}
