\documentclass{article}

% Frequently Asked Questions, Notes, ....
% 01) You can use Czech or English to fill in the template.
% 02) Fill in the metadata section.
%

\usepackage[a-2u]{pdfx}
\usepackage[utf8]{inputenc}
\usepackage{array}
\usepackage[ddmmyyyy]{datetime}
\usepackage{geometry}
\usepackage[skip=10pt]{parskip}

\renewcommand{\dateseparator}{.}
\renewcommand{\baselinestretch}{1.5}

\geometry{margin=2cm}

\def\ResearchProject{Research project}
\def\CompanyProject{Company project}
\def\SoftwareProject{Software project}

% METADATA SECTION START
\def\StudentsFullName{Bc. Ivona Oboňová}
\def\Obor{Computer Science - Software and Data Engineering}
\def\ProjectTitle{Extraction of GroupRule component into standalone service}
% Use "\ResearchProject", "\CompanyProject", \SoftwareProject"
\def\ProjectType{\CompanyProject}
\def\SupervisorFullName{Ing. Pavel Koupil, Ph.D.}
\def\ConsultantsFullName{Bc. Max Kovykov}
\def\ExpectedStart{2025-04-01}
\def\ExpectedEnd{2025-09-01}
% METADATA SECTION END

\title{\ProjectTitle}
\author{\StudentFullName}

\begin{document}

\centerline{\Large \textbf{Proposal of team software project}}
\centerline{ \textbf{Department of Software Engineering}}
\centerline{ \textbf{Faculty of Mathematics and Physics, Charles University}}
\bigskip
{\noindent\begin{tabular*}{\textwidth}{ >{\raggedleft}m{4cm} l}
 {\bf Solvers:} & \StudentsFullName \\
 {\bf Study program:} & \Obor \\
 & \\
 {\bf Project title:} & \ProjectTitle \\
 {\bf Project type:} & \ProjectType \\
 & \\
 {\bf Supervisor:} & \SupervisorFullName \\  
 {\bf Consultants:} & \ConsultantsFullName \\
 & \\
 {\bf Expected start:} & \ExpectedStart \\
 {\bf Expected end:} & \ExpectedEnd \\  
\end{tabular*}}

\section{Introduction}

At \textbf{Avast}, now part of \textbf{Gen Digital}, we specialize in cybersecurity software,
delivering advanced threat detection powered by AI, along with antivirus protection and privacy
solutions such as VPNs and data breach monitoring.

Our backend engineering team plays a central role at Avast, supporting malware analysts, security
researchers, and other experts in enhancing our antivirus products. We process millions of malware
samples daily, managing critical data, while developing and maintaining backend services and APIs.

One of our core systems, \textbf{Appserver}, is a .NET-based monolithic application running on
multiple Windows servers. It stores critical data and provides numerous APIs that analysts rely on
daily to identify and manage malware threats, making it an essential part of our infrastructure.
Among its many components, \textbf{GroupRules} is responsible for managing string-based and
multi-string-based malware definitions and rules.

Over time, Appserver has grown significantly, making it increasingly difficult to navigate,
maintain, and extend with new features. To address these challenges, our team has begun gradually
decoupling the system into a service-based architecture, migrating its logic into independent
services.

This project focuses on migrating the GroupRules component from Appserver to a standalone service.
Transitioning it to a service-based architecture will enhance scalability, maintainability, and
flexibility. By decoupling this component from the monolith, we aim to improve performance and
streamline malware definition management, ensuring a more efficient and future-proof solution.

\section{Project description}

As mentioned in the introduction, we aim to migrate the \textbf{GroupRules} component into a
standalone service, decoupling it from the \textbf{Appserver}, which has grown increasingly
complex, making it difficult to navigate and extend.

Additionally, the GroupRules component is embedded within a complex relational database schema.
The current implementation uses a PostgreSQL database to store data, which has a largely
graph-like structure. As a result, retrieving data in the required format and meeting specific
conditions is both complex and inefficient, further complicating maintenance and updates.

To address these issues, we will not only extract GroupRules into its own service, but also
evaluate whether a graph database would be a better fit for storing and managing string-based and
multi-string-based malware definitions. Given the interconnected nature of the data and its
relationships, graph databases appear to be a promising alternative. As part of this project, we
will assess their feasibility. If a graph database proves to be the optimal choice, we will
identify the best technology and implement GroupRules on top of it. If not, we will continue using
the existing relational database while refining its structure where possible.

Rather than rewriting the entire component, which would introduce significant complexity and risk,
we will focus on rebuilding only its core functionality as an independent service. This approach
allows for gradual integration into the existing infrastructure without disrupting public-facing
APIs and UIs. By maintaining unchanged external interfaces, we ensure a smooth and controlled
transition, enabling early validation of the new architecture.

\section{Form of collaboration}

As the project owner, I will lead the design and planning of this migration, with the support of
our team manager. I will work closely with my team members to ensure a well-structured and
efficient implementation. The outcome will be a scalable and maintainable standalone service that
supports future enhancements while providing a seamless experience for the teams relying on
GroupRules.

\end{document}
